\nonstopmode{}
\documentclass[a4paper]{book}
\usepackage[times,inconsolata,hyper]{Rd}
\usepackage{makeidx}
\usepackage[utf8,latin1]{inputenc}
% \usepackage{graphicx} % @USE GRAPHICX@
\makeindex{}
\begin{document}
\chapter*{}
\begin{center}
{\textbf{\huge Package `lab7'}}
\par\bigskip{\large \today}
\end{center}
\begin{description}
\raggedright{}
\item[Type]\AsIs{Package}
\item[Title]\AsIs{ridge regression}
\item[Version]\AsIs{1.0}
\item[Date]\AsIs{2015-10-21}
\item[Author]\AsIs{Violeta and Yixuan}
\item[Maintainer]\AsIs{yixuan}\email{yixxu916@student.liu.se}\AsIs{}
\item[Description]\AsIs{multiple linear regression model, and for ridge regression.}
\item[License]\AsIs{GPL-2}
\item[Depends]\AsIs{R (>= 3.1.0)}
\item[Suggests]\AsIs{ggplot2, testthat, knitr, rmarkdown, dplyr, nycflights13,
leaps}
\item[Imports]\AsIs{dplyr, ggplot2, nycflights13, leaps}
\item[VignetteBuilder]\AsIs{knitr}
\item[RoxygenNote]\AsIs{5.0.0}
\item[NeedsCompilation]\AsIs{no}
\end{description}
\Rdcontents{\R{} topics documented:}
\inputencoding{utf8}
\HeaderA{coef.ridgereg1}{Print the estimated coefficient values of a ridgereg object}{coef.ridgereg1}
%
\begin{Description}\relax
Print the estimated coefficient values of a ridgereg object
\end{Description}
%
\begin{Usage}
\begin{verbatim}
## S3 method for class 'ridgereg1'
coef(x)
\end{verbatim}
\end{Usage}
%
\begin{Arguments}
\begin{ldescription}
\item[\code{x}] A ridgereg object.
\end{ldescription}
\end{Arguments}
%
\begin{Value}
a coefficients values.
\end{Value}
%
\begin{Examples}
\begin{ExampleCode}
a <- ridgereg1(Petal.Length~Sepal.Width+Sepal.Length, data=iris, lambda=10)
coef(a)
\end{ExampleCode}
\end{Examples}
\inputencoding{utf8}
\HeaderA{predict.ridgereg1}{A method that prints out the fitted values of the ridge regression model, possibly for new data.}{predict.ridgereg1}
%
\begin{Description}\relax
A method that prints out the fitted values of the ridge regression model, possibly for new data.
\end{Description}
%
\begin{Usage}
\begin{verbatim}
## S3 method for class 'ridgereg1'
predict(x, newdata = "default")
\end{verbatim}
\end{Usage}
%
\begin{Arguments}
\begin{ldescription}
\item[\code{x}] A ridgereg object.

\item[\code{newdata}] Optional.
\end{ldescription}
\end{Arguments}
%
\begin{Value}
a numeric vector.
\end{Value}
%
\begin{Examples}
\begin{ExampleCode}
a <- ridgereg1(Petal.Length~Sepal.Width+Sepal.Length, data=iris, lambda=0)
predict(a) # The fitted values
\end{ExampleCode}
\end{Examples}
\inputencoding{utf8}
\HeaderA{print.ridgereg1}{Print a ridgereg object}{print.ridgereg1}
%
\begin{Description}\relax
Print a ridgereg object
\end{Description}
%
\begin{Usage}
\begin{verbatim}
## S3 method for class 'ridgereg1'
print(x)
\end{verbatim}
\end{Usage}
%
\begin{Arguments}
\begin{ldescription}
\item[\code{x}] A ridgereg object.
\end{ldescription}
\end{Arguments}
%
\begin{Value}
coefficients.
\end{Value}
%
\begin{Examples}
\begin{ExampleCode}
a <- ridgereg1(Petal.Length~Sepal.Width+Sepal.Length, data=iris, lambda=0)
print(a)
\end{ExampleCode}
\end{Examples}
\inputencoding{utf8}
\HeaderA{ridgereg1}{A function for ridge regression.}{ridgereg1}
%
\begin{Description}\relax
A function for ridge regression.
\end{Description}
%
\begin{Usage}
\begin{verbatim}
ridgereg1(formula, data, lambda = 0)
\end{verbatim}
\end{Usage}
%
\begin{Arguments}
\begin{ldescription}
\item[\code{formula}] A formula (e.g. y \textasciitilde{} x)

\item[\code{data}] data frame

\item[\code{lambda}] lambda
\end{ldescription}
\end{Arguments}
%
\begin{Value}
ridgereg object.
\end{Value}
%
\begin{Examples}
\begin{ExampleCode}
a <- ridgereg1(Sepal.Length ~ Sepal.Width + Petal.Length, iris, 0)
\end{ExampleCode}
\end{Examples}
\inputencoding{utf8}
\HeaderA{visualize\_airport\_delays}{Handling large datasets with dplyr}{visualize.Rul.airport.Rul.delays}
%
\begin{Description}\relax
Handling large datasets with dplyr
\end{Description}
%
\begin{Usage}
\begin{verbatim}
visualize_airport_delays()
\end{verbatim}
\end{Usage}
%
\begin{Value}
A plot.
\end{Value}
%
\begin{Examples}
\begin{ExampleCode}
library("dplyr")
library("ggplot2")
library("nycflights13")
visualize_airport_delays()
\end{ExampleCode}
\end{Examples}
\printindex{}
\end{document}
